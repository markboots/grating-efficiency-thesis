\documentclass[12pt,bound,twoside,openright,final]{uofsthesis-cs}
%\documentclass[12pt]{uofsthesis-cs}
\usepackage{mathtools}
\mathtoolsset{showonlyrefs}
\usepackage{amssymb}
\usepackage{bbold}
\usepackage{graphicx}
\usepackage{multirow}
\usepackage{tabularx}
\usepackage{topcapt}
\usepackage{booktabs}
\usepackage{enumitem}
\usepackage{hyperref}
\usepackage{listings}
\lstset{breaklines=true,basicstyle=\small}

\graphicspath{{/Users/mboots/Documents/2011/beamteam/thesis/data/}}

\renewcommand{\deg}{^\circ}
\newcommand{\dg}{$^\circ$ } 
\newcommand{\dif}{\mathrm{d}} 

% double-sided, bound:
\newcommand{\aFigureWidth}{\textwidth}
\newcommand{\bFigureWidth}{0.95\textwidth}
\newcommand{\cFigureWidth}{0.8\textwidth}
% non-bound, ETD pdf:


% Hyperlink equation references
\newcommand{\eq}[1]{\hyperref[#1]{\eqref{#1}}}
%\renewcommand{\eqref}[1]{\protect \eqref{#1} }

% What kind of vector font/symbol to use?
\renewcommand{\vec}[1]{\boldsymbol{#1}}

% More vertical space in table cells
\renewcommand{\arraystretch}{1.2}




% Documentation for the uofsthesis-cs class is given in uofsthesis-cs.dvi
% 
% It is recommended that you read the CGSR thesis preparation
% guidelines before proceeding.
% They can be found at http://www.usask.ca/cgsr/thesis/index.htm

%%%%%%%%%%%%%%%%%%%%%%%%%%%%%%%%%%%%%%%%%%%%%%%%%%%%%%%%%%%%%%%%%%%%%%%%%%%%%%
% FRONTMATTER - In this section, specify information to be used to
% typeset the thesis frontmatter.
%%%%%%%%%%%%%%%%%%%%%%%%%%%%%%%%%%%%%%%%%%%%%%%%%%%%%%%%%%%%%%%%%%%%%%%%%%%%%%

% THESIS TITLE
% Specify the title. Set the capitalization how you want it.
\title{Designing and Optimizing Gratings for Soft X-ray Diffraction Efficiency}

% AUTHOR'S NAME
% Your name goes here.
\author{Mark Boots}

% DEGREE SOUGHT.  
% Use \MSc or \PhD here
\degree{\PhD}         

% EXPECTED CONVOCATION DATE
% Should be month/year, e.g. July 2004
\convocationdate{September 2012}


% NAME OF ACADEMIC UNIT
%
% The following two commands allow you to specify the academic unit you belong to.
% This will appear on the title page as
% ``<academic unit> of <department>''.
% So if you are in the division of biomedical engineering you would need to do:
 \department{Physics and Engineering Physics}
% \academicunit{Division}
%
% The default is ``Department of Computer Science'' if these commands
% are not given.
%
% If you are in a discipline other than Computer Science, uncomment the following line and
% specify your discipline/department.  Default is 'Computer Science'.
% \department{If not Computer Science, put the name of your department here}

% If you are not in a department, but say, a division, uncomment the following line.
% \academicunit{Put the type of academic unit you belong to here, e.g. Division, College}


% PERMISSION TO USE ADDRESS
%
% If you are not in Compuer Science you will want to change the
% address on the Permission to Use page.  This is done using the
% \ptuaddress{}.  Example:
%
 \ptuaddress{Head of the Department of Physics and Engineering Physics\\
 Rm 123 Physics Building\\
 116 Science Place\\
 University of Saskatchewan\\
 Saskatoon, Saskatchewan\\
 Canada\\
 S7N 5E2
 }

% ABSTRACT
\abstract{
The \emph{diffraction efficiency} is critical to the speed and sensitivity of grating-based spectroscopy instruments.  This becomes particularly important for soft x-ray instruments, used on material science beamlines at synchrotrons around the world, where the low reflectivity of materials makes it challenging to create efficient optics.

The efficiency of soft x-ray gratings is examined from a rigorous electromagnetic approach using the differential method, adapted for deep gratings using the S-matrix propagation algorithm.  New software is written to provide an open-source implementation with fast performance on cluster computing resources.  Trends in diffraction efficiency are examined as a function of grating materials, coatings, groove geometry, and incidence conditions; these trends are used to provide recommendations for instrument design, including the identification of a new principle of optimal incidence angle.

Efficiency calculations and optimizations are applied to the design of a high-performance soft x-ray emission spectrometer for the REIXS beamline at the Canadian Light Source.  The process produces an innovative design that exploits an efficiency peak in the third diffraction order to offer higher resolution than would otherwise be possible given the space constraints of the machine.  Finally, the spectrometer's actual gratings are measured for diffraction efficiency as a function of wavelength.  Although the real-world efficiencies differ substantially from the nominal calculations, the differences are explained by incorporating real-world effects: geometry errors, groove variation, oxidation, and surface roughness.  A fitting process is proposed to match the calculated to the measured efficiency spectra. The geometry parameters predicted by the fitting process are found to agree exactly with atomic force microscopy (AFM) measurements for all the gratings studied.  Because each grating parameter affects the shape of the efficiency spectrum in a different way, the spectrum can be considered as a unique ``fingerprint'' or ``hash''; we conclude that this might be extended to use efficiency measurements and fitting calculations to characterize grating parameters that are difficult or impossible to measure directly.}

% THESIS ACKNOWLEDGEMENTS -- This can be free-form.
\acknowledgements{
The work described in this thesis represents the collaborative effort of many individuals, as well as the support and encouragement of many others; it has been a privilege to work with and learn from all of them.

In particular, it was my pleasure to work with David Muir along the entire journey from initial design concepts, through seemingly-endless technical problems, to the point of finally using an actual, working emission spectrometer.  His undefeatable commitment and insights made the machine possible.  (We joke that both of us are married to her, but that I never loved her the way he does.)

I am also extremely grateful to my supervisor, Alex Moewes, for providing a deft balance between setting challenges, issuing guidance, and providing the freedom to explore.  The continuous encouragement and light-hearted pressure provided by the members of the material science research group (``\emph{Does it work yet?}'') are also appreciated.  I am grateful to Elder Matias, my supervisor in the controls group at the CLS, for providing the flexibility and understanding that allowed me to finish this degree while simultaneously contributing to the instrumentation and software on the beamline.  Finally, Professor Raymond Spiteri in the Computer Science department provided invaluable help as I was writing the grating software.

The real-world efficiency and AFM measurements, critical to many of the conclusions in this thesis, were only possible through the help of Dr. Eric Gullikson at Beamline 6.3.2 of the Advanced Light Source.  The optimization and fitting calculations used computing resources provided by WestGrid (\url{www.westgrid.ca}) and Compute/Calcul Canada.  I also gratefully acknowledge the funding provided by the NSERC undergraduate and postgraduate scholarship programs.

On a personal level, I am extremely grateful to Kelly Paton for inspiration, mathematical skill, and dedicated editing, as well as Scott Borys for irreplaceable and timely help developing the web application of the grating software.  Last but most of all, I lack appropriate words to thank Joanne Newman for her endless encouragement, support, and love.
}

% THESIS DEDICATION -- Also free-form.  If you don't want a dedication, comment out the following
% line.
%\dedication{I'd like to thank my mom...}

% LIST OF ABBREVIATIONS - Sample  
% If you don't want a list of abbreviations, comment the following 4 lines.
\loa{
\abbrev{AFM}{Atomic Force Microscopy}
\abbrev{API}{Application Programming Interface}
\abbrev{CEM}{Channel Electron Multiplier}
%\abbrev{CF}{Common Scaling Factor}
\abbrev{CIA}{Constant Included Angle}
%\abbrev{EC2}{Elastic Cloud Compute}
\abbrev{EELS}{Electron Energy Loss Spectroscopy}
\abbrev{EUV}{Extreme Ultra-violet}
\abbrev{EXAFS}{Extended Xray Absorption Fine Structure}
%\abbrev{Eff}{Grating Efficiency}
%\abbrev{FFE}{Forces For Efficiency}
\abbrev{GUI}{Graphical User Interface}
\abbrev{HEG}{High Energy Grating}
\abbrev{HPC}{High-performance Computing}
\abbrev{HRHEG}{High-Resolution High Energy Grating}
\abbrev{HRMEG}{High-Resolution Medium Energy Grating}
%\abbrev{IF}{Independent Scaling Factor}
\abbrev{IMP}{Impurity Grating}
\abbrev{IPES}{Inverse Photoelectron Spectroscopy}
\abbrev{LEG}{Low Energy Grating}
\abbrev{MEG}{Medium Energy Grating}
\abbrev{MIM}{Modified Integral Method}
\abbrev{MPI}{Message Passing Interface}
\abbrev{NEXAFS}{Near-edge X-ray Absorption Fine Structure}
\abbrev{PGM,}{Plane Grating Monochromator}
\abbrev{PSD}{Power Spectral Density}
\abbrev{RCW}{Rigorous Coupled Wave}
\abbrev{REIXS}{Resonant Elastic And Inelastic X-ray Scattering}
\abbrev{RIXS}{Resonant Inelastic X-ray Spectroscopy}
%\abbrev{RMS}{Root Mean Square}
\abbrev{SRE}{Stray Radiant Energy}
%\abbrev{SSSC}{University Of Saskatchewan Structural Sciences Centre}
\abbrev{SXE}{Soft X-ray Emission Spectroscopy}
\abbrev{SXS}{Soft X-ray Spectroscopy}
\abbrev{TER.}{Total External Reflection}
\abbrev{TEY}{Total Electron Yield}
\abbrev{TFY}{Total Fluorescence Yield}
\abbrev{TIR}{Total Internal Reflection}
\abbrev{UHV}{Ultra-high Vacuum}
%\abbrev{URF}{United Resolutionary Front}
\abbrev{VLS}{Variable Line Spacing}
\abbrev{XEOL}{X-ray Excited Optical Luminescence}
\abbrev{XPS}{X-ray Photoelectron Spectroscopy}
}

%%%%%%%%%%%%%%%%%%%%%%%%%%%%%%%%%%%%%%%%%%%%%%%%%%%%%%%%%%%%%%%%
% END OF FRONTMATTER SECTION
%%%%%%%%%%%%%%%%%%%%%%%%%%%%%%%%%%%%%%%%%%%%%%%%%%%%%%%%%%%%%%%%

\begin{document}

% Typeset the title page
\maketitle

% Typeset the frontmatter.  
\frontmatter

%%%%%%%%%%%%%%%%%%%%%%%%%%%%%%%%%%%%%%%%%%%%%%%%%%%%%%%%%%%%%%%%
% FIRST CHAPTER OF THESIS BEGINS HERE
%%%%%%%%%%%%%%%%%%%%%%%%%%%%%%%%%%%%%%%%%%%%%%%%%%%%%%%%%%%%%%%%

\chapter{Introduction}

The \emph{diffraction grating} is an optical component that exploits interference from a periodic surface of parallel grooves to control light based on its wavelength.  For almost two hundred years, diffraction gratings have been -- and still are -- at the heart of many instruments responsible for breakthroughs in scientific understanding.  Today, they are are used in astronomy telescopes, chemistry spectrographs, spectrophotometers for life science, and in optics for material science experiments, over a range of wavelengths from the far infrared to soft x-rays.  The sensitivity and acquisition time of these instruments depends on the efficiency of their gratings, i.e., the intensity of useful diffracted light compared to the incident light.  In many cases -- such as Peter Zeeman's famous discovery of energy level splitting in a magnetic field -- improvements in grating efficiency made a previously undetectable effect detectable.

To support and advance these techniques, this project sought to understand and improve the diffraction efficiency of gratings.  While the research happens to be applicable to a wide range of scenarios -- both theoretical and applied, it was focused on a tangible goal: designing and optimizing an innovative soft x-ray emission spectrometer for the REIXS beamline at the Canadian Light Source.  Over the course of the project, we used and created new software tools based on rigorous electromagnetic theory (Chapter 3) to calculate diffraction efficiency.  We applied these tools to understand efficiency trends, and used them to design the optics for the spectrometer.  The software tools, described in Chapter 4, were written to harness high-performance computing resources where available, and have been released to allow other beamline designers to quantify and optimize their own designs.  In examining the trends (Chapter 5), we compared different groove shapes, analyzed the effects of material and geometry parameters, and discovered new principles that can be applied to instrument design, including the principle of optimal incidence.  We used this understanding to design an innovative spectrometer that balances efficiency with high resolving power, and extended its performance by exploiting an efficiency peak in the third diffraction order.  (Chapter 6 describes the design process we used.)

Once the spectrometer gratings were manufactured, we also measured their real-world efficiency and compared it with our calculations (Chapter 7).  The measured efficiencies were very different from the original predictions, but we accounted for the discrepancy based on real-world effects and manufacturing differences: geometry errors, groove variation, surface roughness, and oxidation.  We also discovered a fitting technique that could predict the real grating parameters based on the shape of the measured efficiency curves.  Because each parameter affects the efficiency curves in a different way, we found that we could predict multiple parameters, and confirmed the accuracy of the geometry predictions using atomic force microscopy (AFM) measurements of the actual grooves.  Other parameters like the surface roughness and the oxide thickness are difficult to measure, at least non-destructively; however, the exactness of the fit achieved for all gratings increases our confidence in the theoretical calculations, and suggests it might be possible to use the fitting technique to characterize grating parameters that are infeasible to measure directly.

In most cases of beamline design, the diffraction efficiency is hardly considered, or left up to the grating manufacturer.  It is even more rare to actually {\emph{test}} the gratings for their real efficiency.  The REIXS spectrometer project was successful because we were able to combine rigorous efficiency calculations with ray-tracing predictions of the resolution, and use both to navigate the compromise between these two competing factors.  Commissioning of the real-world spectrometer is still ongoing, but preliminary results (Chapter 8) confirm that our design process produced an effective and useful instrument for material scientists.  In its high efficiency mode, the spectrometer offers competitive resolution and four to six times the throughput of a comparable spectrometer at the Advanced Light Source; it also offers a high resolution mode to push deeper into the electronic structure of new and novel materials.  In addition to this physical instrument, we hope that our efficiency calculation methods and software will be useful in the design of future record-setting soft x-ray beamlines.

% Since thesis chapters are very long and there are a lot of them, it is recommended
% that you put each chapter in a separate .tex file and \input each one of them
% in order.  For example:
%
\input chapter1.tex
\input chapter2.tex
\input chapter3.tex
\input chapter3_5.tex
\input chapter4.tex
\input chapter5.tex
\input chapter6.tex

% ...
%
% The \input command inserts contents of the specified file at the point of the command.

%%%%%%%%%%%%%%%%%%%%%%%%%%%%%%%%%%%%%%%%%%%%%%%%%%%%%%%%%%%%%%%
% SUBSEQUENT CHAPTERS (or \input's)  GO HERE
%%%%%%%%%%%%%%%%%%%%%%%%%%%%%%%%%%%%%%%%%%%%%%%%%%%%%%%%%%%%%%%






%%%%%%%%%%%%%%%%%%%%%%%%%%%%%%%%%%%%%%%%%%%%%%%%%%%%%%%%%%%%%%%%
% The Bibliograpy should go here. BEFORE appendices!
%%%%%%%%%%%%%%%%%%%%%%%%%%%%%%%%%%%%%%%%%%%%%%%%%%%%%%%%%%%%%%%%


% Typeset the Bibliography.  The bibliography style used is "plain".
% Optionally, you can specify the bibliography style to use:
% \uofsbibliography[stylename]{yourbibfile}

\uofsbibliography[abbrv]{referencesFullNames.bib}

% If you are not using bibtex, comment the line above and uncomment
% the line below.  
%Follow the line below with a thebibliography environmentand bibitems.  
% Note: use of bibtex is usually the preferred method.

%\uofsbibliographynobibtex


%%%%%%%%%%%%%%%%%%%%%%%%%%%%%%%%%%%%%%%%%%%%%%%%%%%%%%%%%%%%%%%%%%%%%%%%%
% APPENDICES
%
% Any chapters appearing after the \appendix command get numbered with
% capital letters starting with appendix 'A'.
% New chapters from here on will be called 'Appendix A', 'Appendix B'
% as opposed to 'Chapter 1', 'Chapter 2', etc.
%%%%%%%%%%%%%%%%%%%%%%%%%%%%%%%%%%%%%%%%%%%%%%%%%%%%%%%%%%%%%%%%%%%%%%%%%%

% Activate thesis appendix mode.
\uofsappendix

%\chapter{Sample Appendix}


\end{document}
