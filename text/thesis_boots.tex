\documentclass{uofsthesis-cs}
\usepackage{mathtools}
\mathtoolsset{showonlyrefs}
\usepackage{amssymb}
\usepackage{bbold}
\usepackage{graphicx}
\usepackage{multirow}
\graphicspath{{/Users/mboots/Documents/2011/beamteam/thesis/data/}}

\usepackage{hyperref}

\renewcommand{\deg}{^\circ }
\newcommand{\dg}{$^\circ$ } 
\newcommand{\dif}{\mathrm{d}} 

\newcommand{\eq}[1]{\hyperref[#1]{\eqref{#1}}}

\renewcommand{\vec}[1]{\boldsymbol{#1}}

%\renewcommand{\eqref}[1]{\protect \eqref{#1} }


% Documentation for the uofsthesis-cs class is given in uofsthesis-cs.dvi
% 
% It is recommended that you read the CGSR thesis preparation
% guidelines before proceeding.
% They can be found at http://www.usask.ca/cgsr/thesis/index.htm

%%%%%%%%%%%%%%%%%%%%%%%%%%%%%%%%%%%%%%%%%%%%%%%%%%%%%%%%%%%%%%%%%%%%%%%%%%%%%%
% FRONTMATTER - In this section, specify information to be used to
% typeset the thesis frontmatter.
%%%%%%%%%%%%%%%%%%%%%%%%%%%%%%%%%%%%%%%%%%%%%%%%%%%%%%%%%%%%%%%%%%%%%%%%%%%%%%

% THESIS TITLE
% Specify the title. Set the capitalization how you want it.
\title{Designing and Optimizing Gratings for Soft X-ray Diffraction Efficiency}

% AUTHOR'S NAME
% Your name goes here.
\author{Mark Boots}

% DEGREE SOUGHT.  
% Use \MSc or \PhD here
\degree{\PhD}         

% EXPECTED CONVOCATION DATE
% Should be month/year, e.g. July 2004
\convocationdate{May 2013}


% NAME OF ACADEMIC UNIT
%
% The following two commands allow you to specify the academic unit you belong to.
% This will appear on the title page as
% ``<academic unit> of <department>''.
% So if you are in the division of biomedical engineering you would need to do:
 \department{Physics and Engineering Physics}
% \academicunit{Division}
%
% The default is ``Department of Computer Science'' if these commands
% are not given.
%
% If you are in a discipline other than Computer Science, uncomment the following line and
% specify your discipline/department.  Default is 'Computer Science'.
% \department{If not Computer Science, put the name of your department here}

% If you are not in a department, but say, a division, uncomment the following line.
% \academicunit{Put the type of academic unit you belong to here, e.g. Division, College}


% PERMISSION TO USE ADDRESS
%
% If you are not in Compuer Science you will want to change the
% address on the Permission to Use page.  This is done using the
% \ptuaddress{}.  Example:
%
 \ptuaddress{Head of the Department of Physics and Engineering Physics\\
 Rm 123 Physics Building\\
 116 Science Place\\
 University of Saskatchewan\\
 Saskatoon, Saskatchewan\\
 Canada\\
 S7N 5E2
 }

% ABSTRACT
\abstract{
For almost two hundred years, diffraction gratings have been -- and still are -- at the heart of many instruments responsible for breakthroughs in scientific understanding.  Today, diffraction gratings are used in astronomy telescopes, chemistry spectrographs, spectrophotometers for life science, and in optics for material science experiments, over a range of wavelengths from the far infrared to soft x-rays.  The sensitivity and speed of these instruments depends on the efficiency of gratings, i.e., the intensity of useful diffracted light compared to the incident light.  In some cases -- such as Peter Zeeman's famous discovery of energy level splitting in a magnetic field -- improvements in grating efficiency make a previously-undetectable effect detectable.  

To support these applications, one component of my graduate work has focussed on the diffraction efficiency of gratings.  While this research happens to be applicable to a wide range of scenarios -- both theoretical and applied, I have been working toward a very tangible application: designing and optimizing an innovative soft x-ray emission spectrometer for the REIXS beamline at the Canadian Light Source.  Over the course of this project, I have created software tools to automate efficiency calculations, used these tools to understand efficiency trends, and applied them to the optical design of the spectrometer.  I have also measured the efficiency of real-world gratings and extensively compared these measurements with calculations.

TODO
}

% THESIS ACKNOWLEDGEMENTS -- This can be free-form.
\acknowledgements{
TODO Thanks everybody!
}

% THESIS DEDICATION -- Also free-form.  If you don't want a dedication, comment out the following
% line.
%\dedication{I'd like to thank my mom...}

% LIST OF ABBREVIATIONS - Sample  
% If you don't want a list of abbreviations, comment the following 4 lines.
\loa{
\abbrev{LOF}{List of Figures}
\abbrev{LOT}{List of Tables}
}

%%%%%%%%%%%%%%%%%%%%%%%%%%%%%%%%%%%%%%%%%%%%%%%%%%%%%%%%%%%%%%%%
% END OF FRONTMATTER SECTION
%%%%%%%%%%%%%%%%%%%%%%%%%%%%%%%%%%%%%%%%%%%%%%%%%%%%%%%%%%%%%%%%

\begin{document}

% Typeset the title page
\maketitle

% Typeset the frontmatter.  
\frontmatter

%%%%%%%%%%%%%%%%%%%%%%%%%%%%%%%%%%%%%%%%%%%%%%%%%%%%%%%%%%%%%%%%
% FIRST CHAPTER OF THESIS BEGINS HERE
%%%%%%%%%%%%%%%%%%%%%%%%%%%%%%%%%%%%%%%%%%%%%%%%%%%%%%%%%%%%%%%%

\chapter{Introduction}

blah blah blah

% Since thesis chapters are very long and there are a lot of them, it is recommended
% that you put each chapter in a separate .tex file and \input each one of them
% in order.  For example:
%
\input chapter1.tex
\input chapter2.tex
\input chapter3.tex
\input chapter4.tex
\input chapter5.tex
\input chapter6.tex
\input chapter7.tex

% ...
%
% The \input command inserts contents of the specified file at the point of the command.

%%%%%%%%%%%%%%%%%%%%%%%%%%%%%%%%%%%%%%%%%%%%%%%%%%%%%%%%%%%%%%%
% SUBSEQUENT CHAPTERS (or \input's)  GO HERE
%%%%%%%%%%%%%%%%%%%%%%%%%%%%%%%%%%%%%%%%%%%%%%%%%%%%%%%%%%%%%%%






%%%%%%%%%%%%%%%%%%%%%%%%%%%%%%%%%%%%%%%%%%%%%%%%%%%%%%%%%%%%%%%%
% The Bibliograpy should go here. BEFORE appendices!
%%%%%%%%%%%%%%%%%%%%%%%%%%%%%%%%%%%%%%%%%%%%%%%%%%%%%%%%%%%%%%%%


% Typeset the Bibliography.  The bibliography style used is "plain".
% Optionally, you can specify the bibliography style to use:
% \uofsbibliography[stylename]{yourbibfile}

\uofsbibliography[abbrv]{thesis_msc_boots.bib}

% If you are not using bibtex, comment the line above and uncomment
% the line below.  
%Follow the line below with a thebibliography environmentand bibitems.  
% Note: use of bibtex is usually the preferred method.

%\uofsbibliographynobibtex


%%%%%%%%%%%%%%%%%%%%%%%%%%%%%%%%%%%%%%%%%%%%%%%%%%%%%%%%%%%%%%%%%%%%%%%%%
% APPENDICES
%
% Any chapters appearing after the \appendix command get numbered with
% capital letters starting with appendix 'A'.
% New chapters from here on will be called 'Appendix A', 'Appendix B'
% as opposed to 'Chapter 1', 'Chapter 2', etc.
%%%%%%%%%%%%%%%%%%%%%%%%%%%%%%%%%%%%%%%%%%%%%%%%%%%%%%%%%%%%%%%%%%%%%%%%%%

% Activate thesis appendix mode.
\uofsappendix

\chapter{Sample Appendix}


\end{document}
